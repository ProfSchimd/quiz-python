\documentclass{article}
\usepackage[a4paper, total={7in, 8in}]{geometry}
\usepackage[utf8]{inputenc}
\usepackage{amssymb}
\usepackage{array}
\usepackage{fancyhdr}
\renewcommand{\headrulewidth}{0pt}

\title{Quiz}

\begin{document}

\pagestyle{fancy}
\fancyhf{}
\fancyhf[HC]{
    \begin{tabular}{|m{3.5in}|m{1.25in}|m{1.75in}|}
    \hline 
    &&\\[-9pt]
    Cognome e Nome & Classe  & Data\\[3pt]
    \hline
    \end{tabular}
}

\subsection*{Quiz 1}
Quale è la prima istruzione eseguita da un programma Python?
\begin{itemize}
  \item[$\square$] La prima istruzione della funzione main presente nel file \texttt{.py} invocato.
  \item[$\square$] La prima istruzione della prima funzione definita nel file \texttt{.py} invocato
  \item[$\square$] La prima istruzione presente nel file \texttt{.py} invocato con il comando \texttt{python}
  \item[$\square$] La prima istruzione del metodo \texttt{main} della classe presente nel file \texttt{.py} invocato
\end{itemize}
\subsection*{Quiz 2}
In Python come si accede ai parametri della linea di comando?
\begin{itemize}
  \item[$\square$] Dal modulo \texttt{main} con \texttt{main.argv}
  \item[$\square$] Variabile globale \texttt{argv}
  \item[$\square$] Dal modulo \texttt{sys} con \texttt{sys.argv}
  \item[$\square$] Con un parametro del \texttt{main}
\end{itemize}
\subsection*{Quiz 3}
Indica quale delle seguenti variabili è di tipo \texttt{float}.
\begin{itemize}
  \item[$\square$] \texttt{a = 3.5}
  \item[$\square$] \texttt{a = 7 // 2}
  \item[$\square$] \texttt{a = float(3)}
  \item[$\square$] \texttt{a = 3}
  \item[$\square$] \texttt{a = 3 + 2.5}
  \item[$\square$] \texttt{a = 7 / 2}
\end{itemize}
\subsection*{Quiz 4}
Completa l'istruzione per stampare a video la stringa \texttt{Ciao}\\
\verb|..................|\texttt{('Ciao')}\subsection*{Quiz 5}
Quale è la lista degli argomenti per il comando \texttt{python programma.py Via Roma 33}?
\begin{itemize}
  \item[$\square$] \texttt{['programma.py', 'Via Roma 33']}
  \item[$\square$] \texttt{['programma.py', 'Via', 'Roma', '33']}
  \item[$\square$] \texttt{[Via', 'Roma', '33']}
  \item[$\square$] \texttt{['programma.py', 'Via Roma', '33']}
\end{itemize}
\subsection*{Quiz 6}
Quale di queste function \textbf{non} disponibile di default in Python?
\begin{itemize}
  \item[$\square$] \texttt{array(x)}
  \item[$\square$] \texttt{int(x)}
  \item[$\square$] \texttt{float(x)}
  \item[$\square$] \texttt{str(x)}
\end{itemize}
\subsection*{Quiz 7}
Indica quali operatori \underline{non} sono presenti in Python
\begin{itemize}
  \item[$\square$] Prodotto \texttt{*}
  \item[$\square$] Inverso \texttt{|}
  \item[$\square$] Divisone intera \texttt{//}
  \item[$\square$] Somma \texttt{+}
\end{itemize}
\subsection*{Quiz 8}
Qual'è l'istruzione Python per stampare sulla console?
\begin{itemize}
  \item[$\square$] \texttt{cout}
  \item[$\square$] \texttt{System.out.println}
  \item[$\square$] \texttt{print}
  \item[$\square$] \texttt{console.log}
\end{itemize}


\end{document}
